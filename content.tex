\section{Úvod}

% TODO WRITER: text

Zde vysvětlit problémovou situaci a otázky, které se budou v bakalářské/diplomové práci řešit.

\section{Cíl práce}

% TODO WRITER: text

Smysl a účel, výzkumné otázky.

\section{Metodika zpracování}

% TODO WRITER: text

Cíle, hypotézy/ výzkumné otázky, způsob hledání odpovědí na výzkumné otázky včetně metodiky vlastního výzkumu/šetření, literární rešerše.


\section{Teoretická část}

% TODO WRITER: text

% todo delete poznámky
\section{Herní engine}
\paragraph{}
	Vytvoření hry je  složitý proces, úspěšné dokončení vyžaduje značné úsilí.
	Pokud se podíváme na stav nově vycházejících her, můžeme vidět že i přes roky vývoje od velkých studií najdeme nezměrně případů nedokončených her.
	V nejlepším případě jsou tyto hry podporovány dlouho po samotném vydání , příkladem No mans sky\cite{no_mans_sky}. 
	Jiné jsou jsou zachráněné komunitou, kde komunita sama vydává patch nebo modifikace opravující chyby nebo přidávající  vlastnosti, které zlepšují požitek z dané hry.
	Asi nejznámějším příkladem je Skyrim\cite{skyrim}, kde vývojář Bethesda\cite{bethesda}) je proslulý množstvím problémů, které zanechávají ve svých obrovských hrách.
	Dalo by se téměř říci, že na komunitu modderů spoléhají, jelikož jsou jedna ze společností s oficiální podporou modifikací, dokonce přibalily mody do později vydaných edicí hry Skyrim.
	V nejhorším případě však není zájem ani od komunity ani od vývojáře dále hry podporovat, a tak upadá v zapomnění jako produkt který nikdy nedosáhne své plné kapacity, jedním z novějších případů je Babylons fall\cite{babylons_fall}.
	Není tak divu, že se snažíme nejtěžší části vývojového procesu vyvarovat použitím dostupných nástrojů v podobě knihoven nebo herních enginů.

\paragraph{}
	Herní engine je framework pro podporu vývoje her, který se skládá z knohoven a podpůrných programů vytvořených za tímto účelem.
	Samozřejmě existují samostatné knihovny pro podporu vývoje her, příkladem knihovna Pygame\cite{pygame}, které nám dovolují soustředit se pouze na samotnou hru.
	Jedná se spíše o množství ulehčení pro vývojáře.
	Samostatné knihovny jsou dobré pro rychlé prototypy nebo jednodušší hry, aby jsme však byli schopni vytvořit sofistikované dílo, budeme muset navázat několik knihoven na sebe.
	Mohou zde nastat problémy s kompatibilitou, budeme muset vytvořit propojení, hlídat si aby novější verze knihovny náhodou nerozbila spolupráci a všeobecně knihovnám věnovat množství času které by jsem radši věnovali tvorbě.
	Proto je lepší využít herní engine kde je celý tento proces vyřešen za nás.
	Toto skutečnost si uvědomují i velká vývojový studia a proto můžeme nalézt několik případů vyvíjených herních enginů přímo v daném studiu nebo v sesterských studiích.
	Ne každý engine je však vhodný pro každou hru, tyto enginy jsou často přizpůsobené určitému žánru her a pokusit se v daném enginu vyvíjet jiný typ hry může vést ke značným problémům.
	Také složitost enginů stoupá s jejich schopnostmi.
	Například Frostbite engine\cite{frostbite}, byl popsán jako formule F1, která je krásná a velmi rychlá ale jen zkušený řidič z ní je schopen dostat maximální výkon\cite{frostbite_article}.
	V tomto případě se schopným řidičem zdálo být studio DICE, které pomáhalo ostatním sesterským studiům a ani tak to nebylo v několika případech dostatečné.
	Jedním příkladem za všechny je Mass effect Andromeda\cite{mass_effect_andromeda}, jež je hra která díky svým problémům zastavila vývoj velmi známe a před tímto titulem očekávané série.
	Jedním z hlavních problémů byl totiž přestup na jiný herní engin, než na kterém byla vyvíjena původní trilogie, tímto enginem je právě výše zmíněný frostbite oproti původnímu Unreal enginu\cite{unreal_engine}.

\paragraph{}
	Z výše zmíněných důvodů budeme v této práci také využívat k tvorbě hry enginy, úkolem však je rozhodnout který je pro naše specifické požadavky nejlepší.
	Rozhodně nemůžeme zkoumat všechny možné enginy, proto se omezíme na ty, které jsou dostupné pro indie developery.
	Nemá pro nás totiž cenu zkoumat engine, který je proprietární pro dané studio/společnost, a nemáme tak možnost s ním pracovat.
	Dále by bylo dobré omezit se na nejznámější a nejpoužívanější enginy, protože nemůžeme porovnávat úplně všechny co v době psaní existují a splňují první kritérium.
	Tato činnost by nám zabrala drahocenný čas který je potřeba na vývoj samotné aplikace.
	Mezitím by také dále vycházely nové verze nebo i úplně nové enginy, což by znamenalo navracet se k nim a  a proto by jsme se ocitli v cyklu bez jasné ukončovací podmínky.
	Například stránka ve Wikipedii o herních enginech\cite{list_of_game_engines_wiki} má více jak 150 záznamů.
	Je pravda, že tam můžeme nalézt i záznamy klasifikované jako herní knihovny (pygame\cite{pygame}) a záznamy proprietálních enginů (frostbite\cite{frostbite}).
	To však stále poukazuje na množství enginů, které není možné zahrnout v jedné práci jejíž hlavní přínos není porovnávání herních enginů.

\paragraph{}
	Zaměříme se tedy pouze na několik nejznámějších a nejpoužívanějších případů.
	Dalo by se vybírat enginy dle rychlého vyhledání na internetu, nebo různých "top X" seznamů, máme však k dispozici lepší volbu.
	Naštěstí není problém nalézt data, které alespoň poukazují na nejpoužívanější enginy.
	Tyto data můžeme například najít na Itch.io\cite{itch_io_engines}, které má celou stránku zaměřenou na nejpoužívanější enginy v projektech, které se na itch.io nacházejí.
	Dalším výborným zdrojem dat je Steam, specificky SteamDB\cite{steamdb_engines} který sleduje a zaznamenává statistiky o Steamu.
	Je důležité dodat, že data nikdy nebudou stoprocentně přesná, steamDB má toto upozornění přímo v textu.
	Není neobvyklé aby vznikaly chyby, jsou to spíše odhady založené na datech než absolutní čísla, to nám však k výběru bohatě stačí.
	Dále musíme brát v potaz že existují i další prodejci her, například Epic Games, kteří mají dokonce vlastní unreal engine, který silně podporují.
	U nich se však v době psaní této práce nepodařilo nalézt podobné statistiky a z již předešlých statistik můžeme usuzovat dostačující závěry.
	V nacházejícím texty tedy probereme vybrané herní enginy a nakonec se rozhodneme, který z nich využijeme a uvedeme důvody proč.
	

--todo popsat jednotlivé enginy
\subsection{Unity}

\subsectition{Unreal engine}

\subsection{Game maker}

\subsection{Godot}

\subsection{Construct}

\subsection{RPGMaker}


\paragraph{} godot a proč jsem si ho vybral

	\subsection{poznámky - smazat}
	proč je dobré využít herní engin(zmínit knihovny jako pygame), jaké jsou, následuje krátké highlity enginů, a výsledné zvolení godotu

	https://github.com/pygame/pygame/

	\subsection{Godot}
	co to je(stručně), proč to je (proč byl vybrán  v porovnání s kompeticí [Unity, Unreal, Frostbite,....]), více o  něm (obrázky, výhody do hloubky)

	V oficiální dokumentaci je godot popsán takto:

		\paragraph{}
		Godot Engine is a feature-packed, cross-platform game engine to create 2D and 3D games from a unified interface. It provides a comprehensive set of common tools, so users can focus on making games without having to reinvent the wheel. Games can be exported in one click to a number of platforms, including the major desktop platforms (Linux, macOS, Windows) as well as mobile (Android, iOS) and web-based (HTML5) platforms.

		Godot is completely free and open source under the permissive MIT license. No strings attached, no royalties, nothing. Users' games are theirs, down to the last line of engine code. Godot's development is fully independent and community-driven, empowering users to help shape their engine to match their expectations. It is supported by the Software Freedom Conservancy not-for-profit \cite{godot_introduction} .

		\subsubsection{Poznámky - smazat}
		Godot's key concepts.

		https://docs.godotengine.org/en/stable/getting\_started/introduction/key\_concepts\_overview.html.

		A game is a tree of nodes that you group together into scenes. You can then wire these nodes so they can communicate using signals.

		\begin{itemize}
			\item tree,
			\item node - smallest building blocks organised into trees, godot provides good selection of base nodes,
			\item scene - character, weapon, menu, \ldots,
			\item  signal - when event occur on node, signal is created, used for communication between nodes (observer pattern).
		\end{itemize}

		GDscript is the language

		Godot's design philosophy:
		https://docs.godotengine.org/en/stable/getting\_started/introduction/godot\_design\_philosophy.html
		Object-oriented design and composition

		More about basic concepts
		https://docs.godotengine.org/en/stable/getting\_started/step\_by\_step/index.html

\section{Praktická část}

% TODO WRITER: text


\section{Závěry a doporučení}

% TODO WRITER: text

Kritická diskuze nad výsledky, ke kterým autor dospěl (soulad výsledků  literaturou či předpoklady;
výsledky a okolnosti, které zvláště ovlivnily předkládanou práci atd.). Je vhodné naznačit i případné další
(popř. alternativní) možnosti zkoumání dané problematiky a otevřené problémy pro další studium.

%---------------------------------------------------------end--------todo delete after this
\section{Testovací část}

% TODO delete this testing classes

Vlastní řešení dokládá student zpravidla v několika kapitolách. Podle charakteru práce musí student uvážit, zda informace
netextové povahy (data, tabulky, obrázky atd.) bude uvádět přímo v textu, nebo je zařadí až za celou práci ve formě příloh, či bude kombinovat oba způsoby.
Více podrobností viz Metodické pokyny pro vypracování bakalářských a diplomových prací (zveřejňované formou výnosů děkana)
a v kurzu MES – Metodologický seminář.

	\subsection{Podkapitola}

	Text podkapitoly.

	Následuje ukázka nějakého seznamu:
	\begin{itemize}
		\item fotografie/avatar,
		\item kategorizační štítky,
		\item popisek.
	\end{itemize}

		\subsubsection{Podpodkapitola}

		Lorem ipsum dolor sit amet, consectetur adipiscing elit. Phasellus sit amet ornare diam, id consequat diam.

			\nlparagraph{Paragraf}

			\noindent Ukázka prvního odstavce v paragrafu. Lorem ipsum dolor sit amet, consectetur adipiscing elit. Phasellus sit amet ornare diam, id consequat diam.

			Následuje použití citací: Citace \cite{html_hypertext_markup_language}, \cite{hibernate_docs}, \cite{ddd_quickly}.
			Použití zkratek se rozděluje na první použítí a další použití: první použití \firstac{URL} a další použítí \ac{URL}.

			Použití obrázku je jednoduché, zadáme název souboru, šířku, popisek a zdroj:

			\cntcapfigure{rovnovaha_paka}{8cm}{Páka rovnováhy vzhledu.}{\cite{vizualni_rovnovaha}}

			V případně, že autor je i autor obrázku:

			\cntcapfigure{uhk}{\linewidth}{Toto je UHK.}{[autor]}

			Blok kódu může vypadat takto:

			\begin{codeblock}
				\begin{verbatim}
@Mapper
public interface AccountDao {
  @Select({
    "select *",
    "from " + Account.TABLE_NAME,
    "where email = #{email}"
  })
  Optional<Account> findByEmail(String email);
}
				\end{verbatim}
				\captionsource{Ukázka bloku kódu.}{[autor]}
			\end{codeblock}

			Tabulka zas může vypadat takto:

			\begin{table}[hbt!]
				\captionsource{Ukázková tabulka.}{[autor]}
				\centering
				\begin{tabular}{| l | r | r | r | }
					\hline
					&        psnr &      ssim &      doba  \\
					model &       (db)    &           & gen. (s) \\
					\hline
					bik. int. & 28.3155 & 0.8566 & 0.0322 \\
					nn1000    & 30.1461 & 0.9043 & 0.8109 \\
					nn1001    & 30.0324 & 0.9023 & 0.7486 \\
					nn1002    & \textbf{30.1886} & \textbf{0.9046} & 1.1731 \\
					nn1003    & 30.0390 & 0.9030 & 1.1320 \\
					nn1004    & 24.9772 & 0.7172 & 4.4367 \\
					nn1005    & 26.1629 & 0.8004 & 4.0475 \\
					nn1006    & 27.9129 & 0.8438 & 4.0683 \\
					nn1007    & 27.5834 & 0.8360 & 4.2082 \\
					\hline
				\end{tabular}
			\end{table}

			\newpage

			Další text


